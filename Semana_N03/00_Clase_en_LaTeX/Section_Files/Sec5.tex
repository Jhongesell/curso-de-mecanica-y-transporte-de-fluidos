\begin{frame}{Soluciones aproximadas de la ecuación de Navier Stokes}
\justifying

\begin{figure}[H]
\centering
\includegraphics[scale=0.2]{Section_Files/S3-imagenes-Jhon/0001.png}
\caption{En este capítulo se revisan varias aproximaciones que simplifican la ecuación de Navier-Stokes, incluyendo el flujo reptante, en el que los términos viscosos domian a los términos inerciales. El flujo de lava de un volcán es un ejemplo de flujo trepador: la viscosidad de la roca fundida es tan grande que el número de Reynolds es pequeño aún cuando las escalas de longitud sean grandes.}
\end{figure}

{\tiny Mecánica de fluidos por Yunus A. Cengel, John M. Cimbala, pág. 491}
\end{frame}

%***************************

\begin{frame}{01. Introducción}
\justifying
\begin{figure}[H]
\centering
\includegraphics[scale=0.2]{Section_Files/S3-imagenes-Jhon/0002.png}
\caption{Las soluciones "exactas" comienzan con la ecuación de Navier-Stokes completa, mientras que las soluiciones aproximadas comienzan con una forma simplificada de la ecuación de Navier-Stokes justo desde el principio.}
\end{figure}

{\tiny Mecánica de fluidos por Yunus A. Cengel, John M. Cimbala, pág. 492}
\end{frame}

%***************************

\begin{frame}{02. Ecuaciones adimensionalizadas de movimiento}
\justifying
El objetivo en esta sección es eliminar las dimensiones de las ecuaciones de movimienot, de modo que puedan compararse de manera decuada los órdenes de magnitud de los diversos términos en las ecuaciones. Se comienza con la ecuación de continuidad de flujo incompresible.
\begin{figure}[H]
\centering
\includegraphics[scale=0.2]{Section_Files/S3-imagenes-Jhon/0004.png}
\end{figure}

{\tiny Mecánica de fluidos por Yunus A. Cengel, John M. Cimbala, pág. 493}
\end{frame}

%***************************

\begin{frame}{03. Aproximación de flujo de Stokes}
\justifying
\begin{figure}[H]
\centering
\includegraphics[scale=0.35]{Section_Files/S3-imagenes-Jhon/0022.png}
\caption{El lento flujo de un líquido muy viscoso, en este caso la miel, se clasifica como flujo de Stokes.}
\end{figure}
%\\
{\tiny Mecánica de fluidos por Yunus A. Cengel, John M. Cimbala, pág. 496}
\end{frame}

%***************************

\begin{frame}{Fuerza de arrastre sobre una esfera en flujo de Stokes}
\justifying
\begin{figure}[H]
\centering
\includegraphics[scale=0.35]{Section_Files/S3-imagenes-Jhon/0031.png}
\caption{Para flujo de Stokes sobre un objeto tridimensional la fuerza de arrastre sobre el objeto no depende de la densidad, sino sólo de la velocidad V, alguna longitud característica del objeto $ L $ y la viscosidad del fluido $ \mu $.}
\end{figure}
{\tiny Mecánica de fluidos por Yunus A. Cengel, John M. Cimbala, pág. 499}
\end{frame}

%***************************

\begin{frame}{04. Aproximación para regiones invíscidas de flujo}
\justifying
\begin{figure}[H]
\centering
\includegraphics[scale=0.35]{Section_Files/S3-imagenes-Jhon/0035.png}
\caption{Ecuación de Euler.}
\end{figure}
{\tiny Mecánica de fluidos por Yunus A. Cengel, John M. Cimbala, pág. 501}
\end{frame}

%***************************

\begin{frame}{Derivación de la ecuación de Bernoulli en regiones inviscidas de flujo}
\justifying
\begin{figure}[H]
\centering
\includegraphics[scale=0.35]{Section_Files/S3-imagenes-Jhon/0037.png}
\caption{Ecuación de Euler.}
\end{figure}
{\tiny Mecánica de fluidos por Yunus A. Cengel, John M. Cimbala, pág. 502}
\end{frame}

%***************************

\begin{frame}{05. La aproximación de flujo irrotacional}
\justifying
\begin{figure}[H]
\centering
\includegraphics[scale=0.35]{Section_Files/S3-imagenes-Jhon/0048.png}
\caption{Aproximación irrotacional.}
\end{figure}
{\tiny Mecánica de fluidos por Yunus A. Cengel, John M. Cimbala, pág. 505}
\end{frame}

%***************************

\begin{frame}{Ecuación de continuidad}
\justifying
\begin{figure}[H]
\centering
\includegraphics[scale=0.35]{Section_Files/S3-imagenes-Jhon/0049.png}
\caption{Identidad vectorial.}
\end{figure}
{\tiny Mecánica de fluidos por Yunus A. Cengel, John M. Cimbala, pág. 505}
\end{frame}

%***************************

\begin{frame}{Ecuación de cantidad de movimiento}
\justifying
\begin{figure}[H]
\centering
\includegraphics[scale=0.35]{Section_Files/S3-imagenes-Jhon/0060.png}
%\caption{.}
\end{figure}
{\tiny Mecánica de fluidos por Yunus A. Cengel, John M. Cimbala, pág. 507}
\end{frame}

%***************************

\begin{frame}{Deducción de la ecuación de Bernoulli en regiones irrotacionales de flujo}
\justifying
\begin{figure}[H]
\centering
\includegraphics[scale=0.35]{Section_Files/S3-imagenes-Jhon/0062.png}
%\caption{}
\end{figure}
{\tiny Mecánica de fluidos por Yunus A. Cengel, John M. Cimbala, pág. 507}
\end{frame}

%***************************

\begin{frame}{Regiones irrotacionales bidimensionales de flujo}
\justifying
En regiones irrotacionales de flujo...
\\
{\tiny Mecánica de fluidos por Yunus A. Cengel, John M. Cimbala, pág. 510}
\end{frame}

%***************************

\begin{frame}{Regiones de flujo planar irrotacional}
\justifying
\begin{figure}[H]
\centering
\includegraphics[scale=0.35]{Section_Files/S3-imagenes-Jhon/0072.png}
%\caption{}
\end{figure}
{\tiny Mecánica de fluidos por Yunus A. Cengel, John M. Cimbala, pág. 511}
\end{frame}

%***************************

\begin{frame}{Regiones irrotacionales de flujo aximétrico}
\justifying
\begin{figure}[H]
\centering
\includegraphics[scale=0.35]{Section_Files/S3-imagenes-Jhon/0081.png}
%\caption{}
\end{figure}
{\tiny Mecánica de fluidos por Yunus A. Cengel, John M. Cimbala, pág. 512}
\end{frame}

%***************************

\begin{frame}{Resumen de regiones irrotacionales de flujo bidimensional}
\justifying
Componentes de velocidad para regiones irrotacionales de flujo bidimensional y estacionario de fluido incompresible en términos de función potencial de velocidad y función de corriente en varios sistemas coordenados.
\begin{figure}[H]
\centering
\includegraphics[scale=0.35]{Section_Files/S3-imagenes-Jhon/0088.png}
%\caption{}
\end{figure}
{\tiny Mecánica de fluidos por Yunus A. Cengel, John M. Cimbala, pág. 513}
\end{frame}

%***************************

\begin{frame}{Superposición de flujo en regiones irrotacionales}
\justifying
\begin{figure}[H]
\centering
\includegraphics[scale=0.35]{Section_Files/S3-imagenes-Jhon/0091.png}
\caption{Superposición de dos campos de flujo irrotacional}
\end{figure}
{\tiny Mecánica de fluidos por Yunus A. Cengel, John M. Cimbala, pág. 451}
\end{frame}

%***************************

\begin{frame}{Flujo planares irrotacionales elementales}
\justifying
La superposición permite sumar dos o más soluciones simples de flujo irrotacional, para crear un campo de flujo más complejo (y con la esperanza de ser más significativo físicamente). Por lo tanto, es útil establecer una colección de flujos irrotacionales que sirvan como bloques de la construcción elemental con los que se pueda construir una diversidad de flujos más prácticos...
\\
{\tiny Mecánica de fluidos por Yunus A. Cengel, John M. Cimbala, pág. 514}
\end{frame}


%***************************

\begin{frame}{Flujo irrotacional formados por superposición}
\justifying
Ahora se tiene un conjunto de flujos irrotacionales de bloques de construcción, y se está listo para construir algunos campos de flujo irrotacionales más interesantes mediante la técnica de superposición.
{\tiny Mecánica de fluidos por Yunus A. Cengel, John M. Cimbala, pág. 521}
\end{frame}

%***************************

\begin{frame}{Superposición de un sumidero lineal y un vórtice lineal}
\justifying
\begin{figure}[H]
\centering
\includegraphics[scale=0.35]{Section_Files/S3-imagenes-Jhon/0132.png}
\caption{Superposición.}
\end{figure}
{\tiny Mecánica de fluidos por Yunus A. Cengel, John M. Cimbala, pág. 521}
\end{frame}

%***************************

\begin{frame}{Superposición de un flujo uniforme y un doblete: flujo sobre un cilindro circular}
\justifying
\begin{figure}[H]
\centering
\includegraphics[scale=0.35]{Section_Files/S3-imagenes-Jhon/0137.png}
\caption{Superposición.}
\end{figure}
{\tiny Mecánica de fluidos por Yunus A. Cengel, John M. Cimbala, pág. 522}
\end{frame}

%***************************

\begin{frame}{06. La aproximación de capa límite}
\justifying

\begin{figure}
\centering
\subfloat[Entre la ecuación de Euler (que permite el deslizamiento en las paredes) y la ecuación de Navier-Stokes (que apoya la condición de no deslizamiento) existe un gran vacío.]{
%\label{f:imagen01}
\includegraphics[scale=0.20]{Section_Files/S3-imagenes-Jhon/0163.png}}
\subfloat[La aproximación de capa límite tiende un puente entre ese vacío.]{
%\label{f:imagen02}
\includegraphics[scale=0.20]{Section_Files/S3-imagenes-Jhon/0164.png}}
%\caption{Imagenes}
%\label{f:imagenes01}

\end{figure}

{\tiny Mecánica de fluidos por Yunus A. Cengel, John M. Cimbala, pág. 451}
\end{frame}

%***************************

\begin{frame}{Ecuaciones de la capa límite}
\justifying
\begin{figure}[H]
\centering
\includegraphics[scale=0.25]{Section_Files/S3-imagenes-Jhon/0123.png}
\caption{Sistema coordenado para capa límite de flujo sobre un cuerpo; $ x $ sigue la superficie y, por lo general, se establece en cero el punto de estancamiento frontal del cuerpo, y $ y $ localmente es normal a la superficie en todas partes.}
\end{figure}
{\tiny Mecánica de fluidos por Yunus A. Cengel, John M. Cimbala, pág. 535}
\end{frame}

%***************************

\begin{frame}{El procedimiento de capa límite}
\justifying
\begin{figure}[H]
\centering
\includegraphics[scale=0.25]{Section_Files/S3-imagenes-Jhon/0202.png}
\caption{Resumen del procedimiento de capa límite, para capas límite bidimensionales de flujo estacionario e incompresible en el plano $xy$.}
\end{figure}
{\tiny Mecánica de fluidos por Yunus A. Cengel, John M. Cimbala, pág. 540}
\end{frame}

%***************************

\begin{frame}{Espesor de desplazamiento}
\justifying
\begin{figure}[H]
\centering
\includegraphics[scale=0.25]{Section_Files/S3-imagenes-Jhon/0211.png}
\caption{Espesor de desplazamiento}
\end{figure}
%\\
{\tiny Mecánica de fluidos por Yunus A. Cengel, John M. Cimbala, pág. 544}
\end{frame}

%***************************

\begin{frame}{Espesor de la cantidad de movimiento}
\justifying
\begin{figure}[H]
\centering
\includegraphics[scale=0.25]{Section_Files/S3-imagenes-Jhon/0219.png}
\caption{Espesor de desplazamiento}
\end{figure}
{\tiny Mecánica de fluidos por Yunus A. Cengel, John M. Cimbala, pág. 547}
\end{frame}

%***************************

\begin{frame}{Capa límite turbulenta sobre una placa plana}
\justifying
\begin{figure}[H]
\centering
\includegraphics[scale=0.25]{Section_Files/S3-imagenes-Jhon/0231.png}
\caption{Para una capa límite laminar sobre placa plana, el espesor de desplazamiento es 35.0 por ciento de $ \delta $, y el espesor de la cantidad de movimiento es 13.5 por ciento de $ \delta $. }
\end{figure}
{\tiny Mecánica de fluidos por Yunus A. Cengel, John M. Cimbala, pág. 548}
\end{frame}

%***************************

\begin{frame}{Capas límite con gradientes de presión}
\justifying
\begin{figure}[H]
\centering
\includegraphics[scale=0.25]{Section_Files/S3-imagenes-Jhon/0245.png}
\caption{Las capas limite con gradientes de presión distintos de cero ocurren tanto en flujos externos como en internos: a) capa límite que se forma a lo largo del fuselaje de un avión y hacia la estela, y b) capa límite que crece a lo largo de la pared de un difusor (en abmos casos está exagerado el espesor de la capa límite).}
\end{figure}
{\tiny Mecánica de fluidos por Yunus A. Cengel, John M. Cimbala, pág. 554}
\end{frame}

%***************************

\begin{frame}{Técnica de la integral de la cantidad de movimiento para capas límite}
\begin{figure}[H]
\centering
\includegraphics[scale=0.25]{Section_Files/S3-imagenes-Jhon/0259.png}
\caption{Volumen de control (línea negra punteada gruesa) que se usa en la deducción de la ecuación integral de la cantidad de movimiento (CL se refiere a capa límite.}
\end{figure}
{\tiny Mecánica de fluidos por Yunus A. Cengel, John M. Cimbala, pág. 559}
\end{frame}