\begin{frame}{Soluciones aproximadas de la ecuación de Navier Stokes}
\justifying
Texto
\\
{\tiny Mecánica de fluidos por Yunus A. Cengel, John M. Cimbala, pág. 451}
\end{frame}

%***************************

\begin{frame}{01. Introducción}
\justifying
Texto
\\
{\tiny Mecánica de fluidos por Yunus A. Cengel, John M. Cimbala, pág. 451}
\end{frame}

%***************************

\begin{frame}{02. Ecuaciones adimensionalizadas de movimiento}
\justifying
Texto
\\
{\tiny Mecánica de fluidos por Yunus A. Cengel, John M. Cimbala, pág. 451}
\end{frame}

%***************************

\begin{frame}{03. Aproximaciones de flujo de Stokes}
\justifying
Texto
\\
{\tiny Mecánica de fluidos por Yunus A. Cengel, John M. Cimbala, pág. 451}
\end{frame}

%***************************

\begin{frame}{Fuerza de arrastre sobre una esfera en flujo de Stokes}
\justifying
Texto
\\
{\tiny Mecánica de fluidos por Yunus A. Cengel, John M. Cimbala, pág. 451}
\end{frame}

%***************************

\begin{frame}{04. Aproximación para regiones invíscidas de flujo}
\justifying
Texto
\\
{\tiny Mecánica de fluidos por Yunus A. Cengel, John M. Cimbala, pág. 451}
\end{frame}

%***************************

\begin{frame}{Derivación de la ecuación de Bernoulli en regiones inviscidas de flujo}
\justifying
Texto
\\
{\tiny Mecánica de fluidos por Yunus A. Cengel, John M. Cimbala, pág. 451}
\end{frame}

%***************************

\begin{frame}{05. La aproximación de flujo irrotacional}
\justifying
Texto
\\
{\tiny Mecánica de fluidos por Yunus A. Cengel, John M. Cimbala, pág. 451}
\end{frame}

%***************************

\begin{frame}{Ecuación de continuidad}
\justifying
Texto
\\
{\tiny Mecánica de fluidos por Yunus A. Cengel, John M. Cimbala, pág. 451}
\end{frame}

%***************************

\begin{frame}{Ecuación de cantidad de movimiento}
\justifying
Texto
\\
{\tiny Mecánica de fluidos por Yunus A. Cengel, John M. Cimbala, pág. 451}
\end{frame}

%***************************

\begin{frame}{Deducción de la ecuación de Bernoulli en regiones irrotacionales de flujo}
\justifying
Texto
\\
{\tiny Mecánica de fluidos por Yunus A. Cengel, John M. Cimbala, pág. 451}
\end{frame}

%***************************

\begin{frame}{Regiones de flujo planar irrotacional}
\justifying
Texto
\\
{\tiny Mecánica de fluidos por Yunus A. Cengel, John M. Cimbala, pág. 451}
\end{frame}

%***************************

\begin{frame}{Regiones irrotacionales de flujo aximétrico}
\justifying
Texto
\\
{\tiny Mecánica de fluidos por Yunus A. Cengel, John M. Cimbala, pág. 451}
\end{frame}

%***************************

\begin{frame}{Resumen de regiones irrotacionales de flujo bidimensional}
\justifying
Texto
\\
{\tiny Mecánica de fluidos por Yunus A. Cengel, John M. Cimbala, pág. 451}
\end{frame}

%***************************

\begin{frame}{Superposición de flujo en regiones irrotacionales}
\justifying
Texto
\\
{\tiny Mecánica de fluidos por Yunus A. Cengel, John M. Cimbala, pág. 451}
\end{frame}

%***************************

\begin{frame}{Flujo planares irrotacionales elementales}
\justifying
Texto
\\
{\tiny Mecánica de fluidos por Yunus A. Cengel, John M. Cimbala, pág. 451}
\end{frame}

%***************************

\begin{frame}{Bloque de construcción 1: corriente uniforme}
\justifying
Texto
\\
{\tiny Mecánica de fluidos por Yunus A. Cengel, John M. Cimbala, pág. 451}
\end{frame}

%***************************

\begin{frame}{Bloque de construcción 2: fuente o sumidero lineal}
\justifying
Texto
\\
{\tiny Mecánica de fluidos por Yunus A. Cengel, John M. Cimbala, pág. 451}
\end{frame}

%***************************

\begin{frame}{Bloque de construcción 3: vórtice lineal}
\justifying
Texto
\\
{\tiny Mecánica de fluidos por Yunus A. Cengel, John M. Cimbala, pág. 451}
\end{frame}

%***************************

\begin{frame}{Bloque de construcción 4: doblete}
\justifying
Texto
\\
{\tiny Mecánica de fluidos por Yunus A. Cengel, John M. Cimbala, pág. 451}
\end{frame}

%***************************

\begin{frame}{Flujo irrotacional formados por superposición}
\justifying
Texto
\\
{\tiny Mecánica de fluidos por Yunus A. Cengel, John M. Cimbala, pág. 451}
\end{frame}

%***************************

\begin{frame}{Superposición de un sumidero lineal y un vórtice lineal}
\justifying
Texto
\\
{\tiny Mecánica de fluidos por Yunus A. Cengel, John M. Cimbala, pág. 451}
\end{frame}

%***************************

\begin{frame}{Superposición de un flujo uniforme y un doblete: flujo sobre un cilindro circular}
\justifying
Texto
\\
{\tiny Mecánica de fluidos por Yunus A. Cengel, John M. Cimbala, pág. 451}
\end{frame}

%***************************

\begin{frame}{06. La aproximación de capa límite}
\justifying
Texto
\\
{\tiny Mecánica de fluidos por Yunus A. Cengel, John M. Cimbala, pág. 451}
\end{frame}

%***************************

\begin{frame}{Ecuaciones de la capa límite}
\justifying
Texto
\\
{\tiny Mecánica de fluidos por Yunus A. Cengel, John M. Cimbala, pág. 451}
\end{frame}

%***************************

\begin{frame}{El procedimiento de capa límite}
\justifying
Texto
\\
{\tiny Mecánica de fluidos por Yunus A. Cengel, John M. Cimbala, pág. 451}
\end{frame}

%***************************

\begin{frame}{Espesor de desplazamiento}
\justifying
Texto
\\
{\tiny Mecánica de fluidos por Yunus A. Cengel, John M. Cimbala, pág. 451}
\end{frame}

%***************************

\begin{frame}{Espesor de la cantidad de movimiento}
\justifying
Texto
\\
{\tiny Mecánica de fluidos por Yunus A. Cengel, John M. Cimbala, pág. 451}
\end{frame}

%***************************

\begin{frame}{Capa límite}
\justifying
Texto
\\
{\tiny Mecánica de fluidos por Yunus A. Cengel, John M. Cimbala, pág. 451}
\end{frame}

%***************************

\begin{frame}{Capas límite con gradientes de presión}
\justifying
Texto
\\
{\tiny Mecánica de fluidos por Yunus A. Cengel, John M. Cimbala, pág. 451}
\end{frame}

%***************************

\begin{frame}{Técnica de la integral de la cantidad de movimiento para capas límite}
\justifying
Texto
\\
{\tiny Mecánica de fluidos por Yunus A. Cengel, John M. Cimbala, pág. 451}
\end{frame}